\documentclass{article}
\usepackage[english]{babel}
\usepackage{mathtools}
\usepackage{cite}
\usepackage{hyperref}
\usepackage{url}
\usepackage{xcolor}
\usepackage{bussproofs}

\title{
  A foundational model for types \\
  \large Presentation notes  
}
\author{
  Enríquez Ballester, Adrián 
  \and
  Isasa Martín, Carlos Ignacio
  \and 
  Mata Aguilar, Luis
  \and 
  Bak, Mieczyslaw
}

\begin{document}
\maketitle

\section*{Foundational Proof-Carrying Code}

In 1996, a Gorge Necula's paper \cite{necula:pcc} 
introduces the idea of Proof-Carrying code:

\begin{enumerate}
  \item \textbf{Code producer}: generates an executable 
    together with a proof (certificate) that the program 
    adheres to some safety policy.
  \item \textbf{Code consumer}: receives some untrusted 
    executable with its proof and can validate it before 
    running.
\end{enumerate}

The idea is not about cryptography, but about type systems 
for machine code (i.e. a deductive system defined over 
machine instructions which is proved to guarantee or 
preserve some properties).

Based on that work, Andrew W. Appel, known for being a 
major contributor of the StandardML compiler, starts its 
research in Foundational Proof-Carrying Code, which he 
defines as \cite{appel:fpcc}:

\begin{center}
\textit{
  A framework for mechanical verification of safety 
  properties of machine language...
}
\end{center}

until here it is Proof-Carrying Code

\begin{center}
\textit{
  ...with the smallest possible runtime and verifier.
}
\end{center}

and this last part is the Foundational one. 

The drawback that he sees for Proof-Carrying Code as 
initially described is that it is ad-hoc for each specific 
case, and built-in type rules and lemmas must be a trusted 
part because they are human-verified. Foundational 
Proof-Carrying Code relies on a primitive logic (e.g. high 
order with some arithmetic axioms) which is powerful 
enough to encode the required type system and lemmas. It 
means that they are instead proved in this foundational 
logic. With this, the aim is to make the trusted part as 
small as possible.

At that moment, they chose Twelf for defining the logic and
the required encodings, this is just an example to 
illustrate how it looks like:

\begin{verbatim}
                      tp   : type.
                      tm   : tp -> type.
                      num  : tp.
                      pair : tp -> tp -> tp.
                      
\end{verbatim}

One of the parts that must be modeled within this logic is 
the target machine architecture. The behavior (i.e. 
semantic) and encoding (i.e. syntax) of machine 
instructions must be defined, and they believe that it is
possible for every usual architecture in a similar way 
(i.e. as a step relation $(r,m) \mapsto (r',m')$ where $r$ 
and $r'$ are states of the register bank and $m$ and $m'$ 
of the memory). For example, they encoded the SPARK 
architecture by means of 1035 Twelf LOC for the syntactic 
part, generated with a 151 LOC of a higher level language 
due to redundancies, and 600 Twelf LOC for the semantic 
one. This is an example of an \textit{add} instruction 
encoding:

\begin{align*}
  \mathsf{add}(i, j, k) = \;\;\;\;&\\ 
    \lambda r,m,r',m'.\;&r'(i) = r(j) + r(k) \\
      &\land (\forall x \neq i.\;r'(x) = r(x)) \\
      &\land m' = m \\
\end{align*}

Safety requirements can be specified in the syntax and 
semantics themselves, by making the step relation 
deliberately partial or by making some syntax forbidden 
just by not defining it. This is a dumb example of the 
previous instruction to be not allowed for a certain 
register:

\begin{align*}
  \mathsf{add}(i, j, k) = \;\;\;\;&\\ 
    \lambda r,m,r',m'.\;&r'(i) = r(j) + r(k) \\
      &\land (\forall x \neq i.\;r'(x) = r(x)) \\
      &\land m' = m \\
      &\land i \neq 42 \\
\end{align*}

Now, for proving the adherence to the safety requirements,
an appropriate type system must be defined for the machine
instructions and, as they follow a so called semantic 
approach, it requires the following to be encoded in the 
foundational logic as proofs, not as built-ins:

\begin{itemize}
  \item Type judgements are assigned a truth value.
  \item If a type judgement is true, then it corresponds
    to a safe state.
  \item If the premise judgements are true, then the 
    conclusion judgement must be true.
\end{itemize}

One of the most challenging parts has been to find an 
appropriate model for encoding type systems. Its first 
approach \cite{appel:fpcc:semantic} was to model types as 
sets of values, and model values in a direct way like a 
pair consisting of the memory and a memory address, but 
they encounter some limitations:

\begin{itemize}
  \item They were unable to model mutable fields.
  \item They were unable to model certain kinds of 
    recursive datatype definitions.
\end{itemize}

Their second approach \cite{appel:fpcc:indexed} was to 
model types as sets of pairs $\langle k, v \rangle$ where 
$k$ is an approximation index and $v$ a value. The 
judgement $\langle k, v \rangle \in \tau$ means informally 
that $v$ can be considered to have type $\tau$ for a 
program running for less than $k$ steps. This model solved 
their problem with recursion, but the one with mutable 
fields remained. A PhD thesis of a student of A.W. Appel 
offered later a model which solved also that problem. 

\section*{Interlude}

Ten years later, A.W. Appel says that now it is practical 
to prove safety and correctness with type systems for 
source code instead of machine code and they are 
trustworthy if compiled with a formally verified compiler
\cite{appel:fpcc:compilers}, so he is now involved in 
projects of this kind (e.g. CertiCoq 
\cite{website:certicoq}, CompCert \cite{website:compcert}, 
CertiKOS \cite{website:certikos}).

However, although the results of this research seem to 
have not so much practical interest nowadays, we want to
show how they encoded type systems in a foundational way, 
which is not only applicable to type systems for machine 
code as they show for example with a usual typed lambda 
calculus.

\section*{A foundational model for types}

We are going to show incrementally an example for machine 
instructions with the reasoning of their initial attempt 
to model types within a foundational logic.

The first key insight of FPCC -and of PCC in reality- 
is realizing that you can write type-inference rules
for machine language and states, for example a pair 
projection:

\begin{prooftree}
\AxiomC{$m\vdash x:\tau_1\times\tau_2$}
\UnaryInfC{$m\vdash m(x):\tau_1\wedge m(x+1):\tau_2$}
\end{prooftree}

which means that if $x$ has type $\tau_1\times\tau_2$ 
-meaning a pointer to a pair of those types-, then 
the contents of direction $x$ will be of type $\tau_1$ 
and the contents of direction $x+1$ will be of type 
$\tau_2$. We can even add safety policies into the mix:

\begin{prooftree}
\AxiomC{$m\vdash x:\tau_1\times\tau_2$}
\UnaryInfC{$\mathsf{readable}(x)
  \wedge\mathsf{readable}(x+1)
  \wedge m\vdash m(x):\tau_1\wedge m(x+1):\tau_2$}
\end{prooftree}

The second key insight of FPCC -which really makes it 
foundational-,  is realizing that instead of looking
at this from a syntactical standpoint we can do it 
from a semantical standpoint. Viewing things from a 
syntactic standpoint is significantly harder as they
only allow subject-reduction over a set of axiomatic 
syntactic rules. Using semantics, however, allows us
to have these expressions as lemmas provable from 
their syntactic meaning instead of axioms. Let us 
define $m\vdash x:\tau$ as an application of predicate
$\tau$ on memory $m$ and integer (or address) $x$.
So $m\vdash x:\tau \equiv \tau (m)(x)$. To note the 
change from syntactic to semantic we shall note the 
type judgements as $\models_m x: \tau$. We can now give 
the semantic meaning to the rules written before of:

\begin{align*}
  \mathsf{pair}(\tau_1,\tau_2)(m)(x) \coloneqq
  &\;\mathsf{readable}(x)\\ 
  \wedge&\;\mathsf{readable}(x+1)\\
  \wedge&\tau_1(m)(m(x))\\ 
  \wedge&\tau_2(m)(m(x+1))
\end{align*}

which not only allows us to prove the expressions
that we've already seen but also:

\begin{prooftree}
\AxiomC{$\mathsf{readable}(x)$}
\AxiomC{$\mathsf{readable}(x+1)$}
  \AxiomC{$\models_m m(x): \tau_1$}
  \AxiomC{$\models_m m(x+1) : \tau_2$}
\QuaternaryInfC{$\models_m x: \tau_1\times\tau_2$}
\end{prooftree}

However, all of this poses a problem, what do we 
do if the memory changes? Let us suppose an initial 
state with 
$\models_m r_1: \mathsf{int}\times\mathsf{int}, r_3: 
\mathsf{int}$ and a set of instructions:

\begin{align*}
  &\mathsf{m}(r_2)\leftarrow r_3\\
  &\mathsf{m}(r_2+1)\leftarrow r_3
\end{align*}

How can we prove that 
$\models r_1:_{m'} \mathsf{int}\times\mathsf{int}$ when 
the fact was stated in the initial state of the 
memory but it has changed? We can certainly prove 
theorems of the form:

\begin{prooftree}
\AxiomC{$\models_m v: \mathsf{int}\times\mathsf{int}$}
\AxiomC{$\mathsf{upd}(m,x,y,m')$}
\AxiomC{$x\neq v$}
\AxiomC{$x\neq v+1$}
\QuaternaryInfC{
  $\models_{m'} v: \mathsf{int}\times\mathsf{int}$}
\end{prooftree}

But how can we prove $x\neq v$ and $x\neq v+1$ in a
simple way? The authors of \cite{appel:fpcc:semantic} 
came up with using the heap allocation pointer $a$ as 
a way to check if types are maintained. Giving the 
pair example again, we shall define it as:

\begin{align*}
  \mathsf{pair}(\tau_1,\tau_2)(a,m)(x)&=\\
  x&\in a\wedge x+1\in a\\
  \wedge&\;\mathsf{readable}(x) \\
  \wedge&\;\mathsf{readable}(x+1)\\
  \wedge&\;\tau_1(a, m)(x)\\
  \wedge&\;\tau_2(a, m)(x+1)
\end{align*}

Now, using as an example the register $r_6$, we can 
define what being in $a$ is:

\begin{align*}
    \mathsf{stda}(r,m)(v)=v<r_6
\end{align*}

As the heap expands, new values are stored without 
deleting the old ones. In order to check if the 
registers are kept within this type we must define 
a function to check if they are valid (i.e. 
consistent):

\begin{align*}
  \mathsf{valid}(\tau)=
  &\;\forall a,a',m,v.\;(a \subset a') \implies 
    \tau(a,m)(v) \implies \tau(a',m)(v)\\
  \wedge&\;\forall a,m,m',v.\; (\forall x\;m(x)=m'(x))
    \implies \tau(a,m)(v) \implies \tau(a,m')(v)
\end{align*}

The first part of the definition means that the type 
is invariant under increases of the allocation set
while the second part is that it is invariant under 
the storing of new unallocated variables. With these
new concepts, going back to our examples we can
assert that the register $r_1$ maintains the typing 
$\mathsf{int}\times\mathsf{int}$ as long as that 
type is valid. 

With this kind of reasoning, apart from defining 
specific typing rules for machine instructions as in the 
example, they were also able to model the usual features 
of a type system like type constructors, function types 
and a restricted version of recursive datatypes.

\subsection*{Indexed model for types}

The type system for FPCC that we have discussed has 
some issues, mainly in dealing with recursion. 
In order to solve them, we need a new definition for
type that lets us come with more advanced proofs.

A type is modeled as a set of pairs of the form 
$\langle k, v \rangle$ where $k$ is a nonnegative integer called 
index, $v$ is a value and it is closed under decreasing index 
(i.e.whenever $\langle k, v \rangle$ belongs to a type $\tau$, 
we also have that $\langle j, v \rangle$ belongs to 
$\tau$ for all $j \leq k$).

We write $e :_k \tau$ if $e \rightarrow^j e'$ for some $j < k$ 
and $e'$ in normal form implies 
$\langle k - j, e'\rangle \in \tau$, where $\rightarrow$ 
is the step relation given by the corresponding small step 
semantics.

In each case, the type sets must be defined, but some conventional 
ones which do not depend on a specific use case are for example 

\begin{align*}
  \bot &\equiv \{\} \\ 
  \top &\equiv \{ \langle k, v \rangle\;|\; k \geq 0 \}
\end{align*}

Type environment and states are modeled respectively as mappings 
from variables to types and variables to values. The consistency 
of a state $\sigma$ and environment $\Gamma$ is written 
$\sigma :_k \Gamma$ and defined as $\forall x \in dom(\Gamma)$
we have $\sigma(x) :_k \Gamma(x)$.

The entailment relation $\Gamma \models_k e$ is the semantic 
counterpart of a type judgement and means 
$\sigma(e):_k \alpha$ for all state $\sigma$ consistent with 
$\Gamma$, where $\sigma(e)$ is the result of replacing all the 
free variables of $e$ with their values in $\sigma$. Also, 
$\Gamma \models e : \alpha$ means 
$\Gamma \models e :_k \alpha$ for all $k \geq 0$.

\subsection*{Lambda calculus example}

As an example of type system modeled in this way, they consider 
a lambda calculus with pairs and the constant $0$:

$$
  e \Coloneqq x 
      \;|\; 0 
      \;|\; \langle e_1, e_2 \rangle 
      \;|\; \pi_1(e) 
      \;|\; \pi_2(e)
      \;|\; \lambda x. e 
      \;|\; e_1\;e_2
$$

Values are the constant $0$, a closed abstraction 
$\lambda x.e$ (i.e. no free variables) and a pair of values 
$\langle v_1, v_2 \rangle$.

For the semantics, they consider a pretty standard small step and 
safeness is to not reach a stuck term.

The defined type sets which they define for this use case are,
together with $\bot$ and $\top$:

\begin{align*}
  \mathsf{int} &\equiv \{ \langle k, 0 \rangle \;|\; 
    \forall k.\;k \geq 0 \} \\
  \tau_1 \times \tau_2 &\equiv 
    \{ \langle k, \langle v_1, v_2 \rangle \rangle \;|\; 
          \forall j.\;j < k \land
          \langle j, v_1\rangle \in \tau_1  \land 
          \langle j, v_2 \rangle \in \tau_2
    \} \\
  \alpha \rightarrow \tau &\equiv 
    \{ \langle k, \lambda x. e \rangle \;|\; 
        \forall j.\;j < k \land
        \langle j, v \rangle \in \alpha \implies
          e[v/x] :_j \tau
    \} \\ 
  \mu F &\equiv 
    \{ \langle k, v \rangle \;|\; \langle k, v \rangle \in 
      F^{k + 1}(\bot)
    \}
\end{align*}

The relation $\models e : \alpha$ models safety trivially from 
its definition, and the rules which correspond to the axioms 
of variables and the constant $0$ can also be trivially proved 
from the definitions of consistency and the type set 
$\mathsf{int}$.

The same happens in order to prove that the type sets product 
and abstraction are also valid types (i.e. they satisfy the 
closed under decreasing index property).

With that, the remaining rules can also be proved. For example,
the idea of the proof for the application rule consists on 
the fact that, for a given $k$, the left subterm either does not 
reach a normal form yet or it reduces to a lambda expression. By 
checking then the result of beta reducing it and the corresponding 
type set definitions, the rule follows.

(Show a slide with the typing rules and comment that they are 
written with their entailment relation syntax but in fact 
they are defined within the logic they use).

This new approach allows them to model more advanced features 
such as quantified types and type equalities.

\section*{Conclusion}

Foundational Proof Carrying Code seems to be out of interest 
nowadays, but it motivated a research about modeling type systems 
in a foundational way.

In that research, to model types directly as sets of values was
not enough for some features such as contravariant recursive
definitions and mutable fields.

A more sophisticated model allowed them to better model recursion, 
and further research seems to have solved even the problem they 
had about mutable fields.

These results may help in studying type systems from the point of 
view of foundational mathematics and also where the introduction 
of rules must be correct within a logic in a machine-checkable 
way.

This is a list of some of the features successfully modeled within 
their research:

\begin{itemize}
  \item Usual features of a functional language.
  \item Usual features of an imperative and OO language.
  \item Specific type systems for machine instructions.
  \item Recursive datatype definitions.
  \item Function types (i.e. first-class functions, continuations 
    and closures).
  \item Universal and existential quantification.
  \item Type equality (i.e. $\tau_1 \sim \tau_2$).
\end{itemize}


\bibliography{refs}{}
\bibliographystyle{plain}

\end{document}
